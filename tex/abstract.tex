
\section{Abstract}

At the end of the 2nd millennia, the Clay institute of mathematics put together a list of seven problems that would be the most important and hard problems in mathematics to solve for the next millennia. 
The problems on that list is now the seven most famous problems in the world of mathematics, and for good reasons. 
Three years later one of the problems, the Poincaré conjecture, was solved, and this problem will be the focus of this talk. 

The Poincaré conjecture (really the Poincaré theorem, but the name stuck for historical reasons) is a very fundamental problem in the field of topology, and was the last piece in the puzzle of understanding which spaces are actually spheres in any dimension. 
We will cover some basic introductory differential topology, the statement of the conjecture, the general outline of the proof and finally its generalizations and related problems.