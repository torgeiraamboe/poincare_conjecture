

\section{Preliminaries}

\begin{itemize}
    \item Define homology groups
    \item Define the fundamental group
\end{itemize}

The Poincaré conjecture very roughly states that every 3-dimensional shape that has no holes, has a defined inside and outside and no boundary, must be the 3-dimensional sphere. 
It is a bit more complicated than this, but this is essentially the statement. 
So what is a 3-dimensional shape, and how can we describe these properties that we mentioned using proper mathematics? 

Let's start with shapes. 
In mathematics, and in particular the field of topology, shapes are described by manifolds. 
An $n$-dimensional manifold is a shape that looks locally like $n$-dimensional Euclidean space. 
An example is the surface of the earth. 
It is locally flat, meaning that the area you see around you looks a bit like just a flat plane, i.e. 2-dimensional Euclidean space. 
There might be some bumps and cavities, but these can be ignored. 

\begin{definition}
An $n$-dimensional manifold is a (second countable Hausdorff) topological space $M$ such that for every point $p\in M$, there is an open neighborhood $U$ of $p$ and a homeomorphism $\phi:U\longrightarrow V$, where $V$ is an open set in $\R^n$.
\end{definition}

Here a homeomorphism means that $\phi$ is a continuous isomorphism where its inverse is also continuous. 

The next thing we need is the notion of homology. 
I will not cover homology in full detail here.
It is a really important gadget, but it takes time to cover it properly. 
That said I have added the definition below.  

\begin{definition}
Let $M$ be a manifold. 
We define the n'th singular homology group of $M$ to be the quotient  

$$H_n(M) = \frac{Z_n(M)}{B_n(M)} = \frac{Ker(\partial_n)}{Im(\partial_{n+1})},$$

where $Z_n(M)$ is the group of $n$-cycles and $B_n(X)$ is the group of $n$-boundaries. 
\end{definition}



