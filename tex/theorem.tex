

\section{The statement of the conjecture}

\begin{itemize}
    \item Add Poincaré's first attempt at homology spheres being spheres
    \item Add the original statement
    \item Add the modern statement
\end{itemize}

It seems like an arbitrary simple question related to one of part of a classification of manifolds, so why is the question important? 

One way to think about it is the following. 
In ancient Egypt people were able to figure out using mathematics that the earth was actually not flat, but it was spherical. 
This was hundreds of years before any human could travel outside the earth and look at the global structure, and not just the local ``flat'' one. 
This shows that we can use local information to prove how the global information must be, at least in the case of the earth. 
Now, ask the same question but a bit bigger. 
Can humans ever (theoretically) confirm the shape of the universe without traveling outside it to view the global structure? 
The Poincaré conjecture can be thought of as a mathematical gadget that gives us a partial answer to this. 
The most likely shapes for the universe is either a sphere or a flat manifold (torus, cylinder), and if true, the Poincaré conjecture gives us a condition to answer this. 
It gives us a (theoretically) testable procedure that would allow us to say for certain if the universe is a sphere or not. 
Just simply fly everywhere in the universe with a rope to form loops and try to pull them in. If all loops can be pulled in, then we must live in a spherical universe.   


\section{The solution}

Let $M$ be a closed 3-manifold. 
Any Riemannian metric with constant Ricci curvature has constant sectional curvature. 
By the Killing-Hopf theorem, any connected Riemannian manifold with constant positive curvature is homeomorphic to a quotient of its universal cover, $S^3$, with the linear free action of its fundamental group. 
If we assume $M$ is simply connected and closed, like in the Poincare conjecture, a constant curvature must mean a constant positive one. 
Hence a possible solution to the Poincare conjecture comes from solving the following question; 
How can we construct a Riemannian metric on $M$ with constant Ricci curvature?  


\subsection{Ricci flow}

\textbf{Slogan:} Ricci flow is a procedure for making a manifold ``rounder'' over time. 

The Ricci flow equation is the following differential equation

$$\frac{\partial g(t)}{\partial t} = -2 Ric(g(t))$$

where $Ric(g(t))$ is the Ricci curvature of the metric $g(t)$. 
Fixed points of this equation are the metrics with constant curvature. 
A key insight by Hamilton was that if the Ricci curvature was always positive, i.e. $Ric(g(t))> 0$ then the manifold $M$ would become a point in finite time. 
If we introduce a lower bound for the diameter of the manifold, we would get constant positive curvature everywhere, and hence have a sphere by the discussion earlier. 
Hence we can define the Ricci flow on an interval $[0, T)$ where $T$ is dependent on the initial metric. 

\begin{example}
Let $M$ be the sphere with radius $r$ in $\mathbb{R}^3$. 
Then $g_{ij} = r^2\hat{g}_{ij}$ where $\hat{g}_{ij}$ is the metric for the unit sphere $S^3$. 
The Ricci tensor $Ric_{ij} = 2\hat{g}_{ij}$ is independent of $r$. 
Hence the Ricci flow becomes $\frac{dr^2}{dt} = -2\cdot 2$, which has a solution

$$r^2(t) = r^2(0)-2\cdot2t$$

Hence $M$ collapses to a point in finite time. 
\end{example}

If however the initial metric is not positive, and we have a more complicated topology, for example in a complicated connected sum, we can get what we call finite time singularities forced by the topology. 

\begin{example}
The simplest example comes from letting $(M, g)$ be a closed Riemannian Einstein manifold. 
A manifold is called an Einstein manifold if there exists a $\lambda$ such that $Ric(g(t)) = \lambda g(t)$. 
Then the equation $g_t = (1-2\lambda t)g$ is a Ricci flow with $g_0 = g$. 
We se this because 

$$\frac{\partial g_t}{\partial t}=-2\lambda g = -2 Ric(g(t)) = -2Ric(g_t(t)).$$

If we let $\lambda> 0$ then $1-2\lambda t < 1$ when $t>0$. 
Hence we need $t<\frac{1}{2}\lambda$ in order for $g_t$ to be a Riemannian metric. 
This means we get a singularity in finite time. 
\end{example}

To solve this problem, i.e. having these nasty singularities, Hamilton introduced Ricci flow with surgery. 


\subsection{Ricci flow with surgery}

\textbf{Slogan:} Ricci flow with surgery is like Ricci flow, except we are allowed to surgically cut away finite time singularities during the process. 

If we have a finite amount of singularities, we can ``split'' $[0, T)$ into a finite number of disjoint intervals $[0, T_1), [T_1, T_2), \ldots, [T_{n-1}, T_n)$. 
Now the manifold changes with the Ricci flow discontinuously only finitely many times.

The key insight of Perelman was that if these singularities appear in finite time they must look like spheres or cylinders, i.e. $\mathbb{R}\times S^2$ or $S^3.$  
These manifolds we can easily cut away and continue the Ricci flow on the components we get. 
Perelman proved that this Ricci flow, where we are allowed to surgically remove singularities, also converge to a point in finite time for simply connected compact 3-manifolds. 

In fact he proved something even stronger. 
We don't even need simply connected, only that the fundamental group of the manifold is a free product of finite groups and infinite cyclic groups. 
This condition essentially means that if we decompose our manifold $M$ into its prime decomposition, then none of the components are weird and difficult manifolds. 
This means that all the geometric components of $M$ have Thurston geometries based on the $\mathbb{R}\times S^2$ and $S^3$ geometries. 
Perelman also used an extended method of infinite time to prove the more general Thurston geometrization theorem, which has the Poincaré conjecture as a consequence as well. 

So, lets devise a cook-book strategy for showing that any simply connected closed 3-manifold is actually the 3-sphere. 
\begin{enumerate}
    \item Start by running Ricci flow on $M$. This will produce several spheres, connected by hair-like strands. 
    \item Cut these strands off and continue the Ricci flow on the components. 
    \item Repeat this until we (by diameter restrictions) only have disconnected spheres left. 
    \item The connected sum of these is a sphere, and since the original manifold was connected, $M$ also has to have been a sphere all along.  
\end{enumerate}


\section{Generalizations and related problems}

Some related problems of this theorem were actually proven before the Poincaré conjecture itself. 
The same statement was actually proven in all dimensions except for $n=4$, i.e. it was known that every homotopy $n$-sphere is homeomorphic to $S^n$ for $0< n\neq 4$. 
For $n=1, 2$ these are quite easy problems, but there was little belief in that this should hold for $n\geq 5$. 
The topology community was quite surprised when Stephen Smale proved it to hold for $n\geq 5$ in 1961. 
The case $n=4$ was proven in 1982 by Michael Freedman. 

We could also try to do the same statement in other categories of manifolds. 

So, fix a category $\mathcal{C}$  of manifolds, i.e. $TopMan$, $Diff$ or $PLMan$. 
The generalized Poincaré conjecture states that for every $n$, any object in $\mathcal{C}$ homotopy equivalent to the $n$-sphere is also isomorphic, in $\mathcal{C}$, to the $n$-sphere. 
The normal Poincaré conjecture was the last piece of this theorem for $\mathcal{C} = TopMan$, hence the generalized theorem is true in this setting. 

For smooth manifolds the theorem is false in general, due to Milnor proving the existence of exotic spheres when $n=7$. 
It is however true for $n = 1, 2, 3, 5$ and $6$ . 
It is still an open problem weather it holds in dimension $4$. 
This is often called the smooth Poincaré conjecture. 
It is however believed to be false by many in the field. 
For $\mathcal{C} = PLMan$ the case in dimension $4$ is proven to be equivalent to the differential case, and is hence also still open. 
It is however true for all $n\neq 4$.